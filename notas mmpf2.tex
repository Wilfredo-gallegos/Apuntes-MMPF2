\documentclass{article}
% Language setting
% Replace `english' with e.g. `spanish' to change the document language
\usepackage[spanish]{babel}

% Set page size and margins
% Replace `letterpaper' with`a4paper' for UK/EU standard size
\usepackage[letterpaper,top=2cm,bottom=2cm,left=1cm,right=1cm,marginparwidth=1.75cm]{geometry}

% Useful packages
\def\alignboxed#1&#2\endalignboxed{\rlap{\boxed{#1#2}}\hphantom{#1\mkern6mu}
}
\usepackage{bigints}
\usepackage{blindtext}
\usepackage{verbatim}
\usepackage{amsmath}
\usepackage{amssymb}
\usepackage[makeroom]{cancel}
\usepackage{amsthm}
\usepackage{amsfonts}
\theoremstyle{definition}
\newtheorem{definition}{Definición}[section]
\newtheorem{theorem}{Teorema}[section]
\newtheorem{corollary}{Corolario}[theorem]
\newtheorem{lemma}{Lema}[section]
\usepackage{etoolbox}
\usepackage{graphicx}
\usepackage{upgreek}
\newcommand{\bigslant}[2]{{\raisebox{.2em}{$#1$}\left/\raisebox{-.2em}{$#2$}\right.}}
\usepackage{xcolor}
\usepackage{mathtools, bm}
\usepackage{multicol}
\usepackage{tikz}
\makeatletter
\newcommand\mathcircled[1]{%
  \mathpalette\@mathcircled{#1}%
}
\newcommand\@mathcircled[2]{%
  \tikz[baseline=(math.base)] \node[draw,circle,inner sep=1pt] (math) {$\m@th#1#2$};%
}
%\setlength{\columnseprule}{1pt}
\usepackage[colorlinks=true, allcolors=blue]{hyperref}

\title{Metodos matemáticos 2}
\author{Wilfredo Gallegos}

\begin{document}
\maketitle

\textbf{viernes 7 de julio}
\section{Función Gamma}
\textbf{Límite al infinito(Euler)}

\[\begin{array}{rl}
	\Gamma (z)\equiv & \lim\limits_{n\to \infty} \frac{1\cdot 2\cdot 3\cdot\cdot\cdot n}{z(z+1)(z+2)\cdot\cdot\cdot (z+n)}n^z,\ \ \ z\in\mathbb{Z}^{+}\ o\ z\in\mathbb{C}\\
	\Gamma(z+1)=& \lim\limits_{n\to \infty} \frac{1\cdot 2\cdot 3\cdot\cdot\cdot n}{(z+1)(z+2)\cdot\cdot\cdot (z+n+1)}n^{z+1}\\
	=& z\cdot\Gamma (z)= \lim\limits_{n\to \infty}\frac{nz}{z+n+1}\Gamma (z)=\Gamma (z+1)=z\cdot\Gamma (z)
\end{array}\]
Aplicando lo anterior a z=1,2,3...n

\[\begin{array}{rl}

	\Gamma (1)=& \lim\limits_{n\to \infty} \frac{1\cdot 2\cdot 3\cdot\cdot\cdot n}{(1)(2)\cdot\cdot\cdot (n+1)}n^z= \lim\limits_{n\to \infty}\frac{n}{n+1}=1\\
	\Gamma (2)=& z\cdot \Gamma (z)= \Gamma (z+1)=\Gamma (1+1)=1\\
	\Gamma (3)=& 2\cdot \Gamma (1)= \Gamma (2+1)=2\cdot 1\\
	\Gamma (4)=& 3\cdot \Gamma (3)= \Gamma (3+1)=3\cdot 2\cdot 1\\
	.\\
	.\\
	.\\
	\Gamma (n)=&(n-1)!

\end{array}\]
\textbf{Itegral Definida(Integral de Euler)}
\[\Gamma (z)=\int^{\infty}_0 e^{-t}t^{z-1}dt,\ \ \ Re(z)>0\]
\textbf{\textcolor{red}{Ej.}} 

Como aparecen en física
\[\Gamma (z)= 2\int^{\infty}_0 e^{-t^2}t^{z+1}dt\ \ \ o \ \ \ \Gamma (z) = \int^1_0\left[ln(\frac{1}{t})\right]^{z-1}dt\]

Si $z=\frac{1}{2}\Rightarrow \Gamma (\frac{1}{2}) = \sqrt{\pi}$ es integral error de Gauss
\[F(z,n)=\int^n_0\left(1-\frac{t}{n}\right)^nt^{z-1}dt,\ \ \ \text{Re(z)}>\text{0 con n entero positivo}\ni\]
\[e^{-t}=\lim\limits_{n\to \infty}\left(1-\frac{t}{n}\right)^n\]
\[\Rightarrow F(z,n)=\int^n_0\left(1-\frac{t}{n}\right)^nt^{z-1}dt = \int^1_0 (1-u)^n(un)^{z-1}ndu\ni u=\frac{t}{n}\]
\[\frac{F(z,n)}{n^z}=\int^1_0 (1-u)^n(u)^{z-1}du\]

Ahora por integración por partes usando $u=(1-u)^n$, $du=n(1-u)^{n-1}du$, $v=\frac{u^z}{z}$, $dv=u^{z-1}du$ tenemos
\[
\begin{array}{rl}
	\frac{F(z,n)}{n^z}=& (1-u)^n \left.\frac{u^z}{z}\right|^1_0-\int^1_0\frac{u^z}{z} n(1-u)^{n-1}du\\
	F(z,n)=& n^z\cdot \frac{n(n-1)\cdot\cdot\cdot 1}{z(z+1)\cdot\cdot\cdot (z+n-1)}=\int^1_0u^{z+n-1}du\\
	=& \frac{1\cdot 2\cdot\cdot\cdot n}{z(z+1)\cdot\cdot\cdot (z+n)}n^z\Rightarrow\lim\limits_{n\to \infty}F(z,n)=\Gamma (z)
\end{array}\]

\textbf{Producto infinito(Weierstrass)}
\[\frac{1}{\Gamma (z)}= z e^{\delta z} \prod^{\infty}_{n=1}\left(1+\frac{z}{n}\right)e^{-\frac{z}{n}}\]
\begin{center}
	$\delta$ := constante de Euler-Mascheron
	
	$\delta$ := 0.5772156619
\end{center}

\textbf{Ecuación estadística de Maxwell-Boltzmann}
\[\begin{array}{lcl}
	&& \text{K es la constante de Boltzmann}\\
	e^{-E/KT} && \text{T es la temperatura absoluta}\\
	&& \text{E(energía): estado de energía ocupada}

\end{array}\]

probabilidad de estar en estado de energía es $Y_{kt}=\beta$

\[\begin{array}{lcl}
	P(E)=Ce^{-\beta E} && \text{Para un gas idel sin estructura}\\
	n(E)dE && n(E)^{1/2}\\
	1=C\int n(E) e^{-\beta E}dE && \text{E=energía cinética}
\end{array}\]
\par\noindent\rule{\textwidth}{0.5pt}
\[
\begin{array}{rlr}
	1=& c\int^{\infty}_0 E^{1/2}e^{-\beta E}dE=\frac{C\Gamma (3/2)}{\beta^{3/2}},& \beta E = T\Rightarrow dE=\frac{dt}{\beta}\\
	1=& C\int^{\infty}_0 e^{-t}\left(\frac{t}{\beta}\right)^{1/2}\frac{1}{\beta} dt\\
	&= C \int^{\infty}_0 e^{-t} t^{3/2-1}dt\cdot\frac{1}{\beta^{3/2}}\\
	=& \frac{C\Gamma (3/2)}{\beta^{3/2}}\\
	=& c\cdot\frac{\sqrt{\pi}}{2\cdot\beta^{3/2}}
\end{array}\]
\[\therefore C= \frac{2\cdot \beta^{3/2}}{\sqrt{\pi}}\]
\newpage

\textbf{lunes 10 de julio}\\
\textbf{Relaciones de funciones}

\[\Gamma (z+1) = z\Gamma (z)\]
\textbf{Fomula de reflexión}

\[\Gamma (z)\Gamma (1-z) = \frac{\pi}{sen(z\pi)}\]
\textbf{Fórmuláde duplicación de Legendre}

\[\Gamma (1+z)\Gamma (z+\frac{1}{2}) = 2^{-2z}\sqrt{\pi}\Gamma (2z+1) \]
\textbf{Notacíón doble factorial: }
\[
\begin{array}{rl}
	(2n+1)!!=& 1\cdot 3\cdot 5\cdot\cdot\cdot (2n+1)=\frac{(2n+1)!}{2^nn}\\
	(2n)!!=& 2\cdot 4\cdot 6\cdot\cdot\cdot (2n)=2^nn!\\
	(-1)!!=& 1
\end{array}
\]
\textbf{PROPIEDADES ANALíTICAS}

$[\Gamma (z)]^{-1}$ tiene singularidades en $z=0,-1,-2...$ y no tiene cuando  $z=-1,2,3...$ y no tiene ceros en el plano complejo finito positivo.\\
El residuo $R_n$ cuando $z=-n$ donde n es un entero mayor o igual a cero.

\[
\begin{array}{rl}
	R_n=&\lim\limits_{n\to 0} (\epsilon\Gamma (-n+\epsilon))\\
	=& \lim\limits_{n\to 0} \frac{\epsilon\Gamma (-n+1+\epsilon)}{-n+\epsilon}\\
	=& \lim\limits_{n\to 0} \frac{\epsilon\Gamma (-n+2+\epsilon)}{(-n+\epsilon)(-n+1+\epsilon)}\\
	=&\lim\limits_{n\to 0} \frac{\epsilon\Gamma (1+\epsilon)}{(-n+\epsilon)\cdot\cdot\cdot(\epsilon))}\\
	=& \frac{(-1)^n}{n!}, \text{el residuo alterna signos en } z=-n
\end{array}
\]

\textbf{Integral de Schaefli}
\[\int_Ce^{-t}t^vdt=(e^{2\pi iv})\Gamma (v+1)\]
donde C es el contorno
\begin{center}
\includegraphics[scale=0.4]{../grafico 1.png} 
\end{center}

Esta integral es útil cuando V no es entero.\\
\textbf{Notación factorial:}
\[\prod (z)=z!=\Gamma (z+1)\]
\textbf{Función Digamma}

\[
\begin{array}{rl}
	\ln(\Gamma (z+1))=& z\cdot\Gamma (z)\\
	=& \lim\limits_{n\to \infty}\frac{z}{z} \ln\left[\frac{n!\cdot n^z}{(z+1)(z_2)\cdot\cdot\cdot (z+n)}\right]\\ \\
	\frac{d}{dz} \ln(\Gamma (z+1))=& \frac{d}{dz}\lim\limits_{n\to \infty}\left[\ln(n!)+z\cdot \ln(n)-\ln(z+1)-\ln(z+2)-...\ln(z+n)\right]\\ \\
	\frac{d}{dz}\ln (\Gamma (z+1))\equiv \underbrace{\varphi (z+1)}_{\parbox{3.5em}{\footnotesize Función digamma}}=& \lim\limits_{n\to \infty}\left( \ln (n)-\frac{1}{z+1}-\frac{1}{z+2}\cdot\cdot\cdot -\frac{1}{z+n}\right)\\ \\
	\varphi (z+1)=& \frac{[\Gamma (z+1)]'}{\Gamma (z+1)}
\end{array}\]
Si sumamos y restamos un número armónico
\[H_n=\sum^n_{m=1}\frac{1}{m}\]
\[\varphi (z+1) =\lim\limits_{n\to \infty} \left[\underbrace{(\ln (n)-H_n)}_{\parbox{4.5em}{\footnotesize $-\delta$\text{ cuando } $n\to\infty$}}-\sum\left(\frac{1}{z+m}-\frac{1}{m}\right)\right]\]
\[\Rightarrow \varphi (z+1) = -\delta +\sum^{\infty}_{m=1}\frac{z}{m(m+z)},\ \ \ \text{si z=0}\Rightarrow\varphi (1)=-\delta\]
para $n>0$, n entero.
\[
\begin{array}{rl}
	\varphi (n+1) =& -\delta +\sum\limits^{n}_{m=1}\frac{1}{m}\\
	\\
	=& -\delta+H_n
\end{array}
\]
\textbf{Función Poligamma}

Cuando se deriva muchas veces la función digamma
\[
\begin{array}{rl}
	\varphi^{(m)} (z+1) \equiv& \frac{d^{m+1}}{dz^{m+1}}\ln\Gamma (z+1)\\
	\\
	\varphi^{(m)} (z+1)=& (-1)^{m+1}m!\cdot \sum\limits^{\infty}_{n=1}\frac{1}{(z+n)^{m+1}},\ \ \ m=1,2,3...
\end{array}
\]
Si z=0 la función se conoce como zeta de Riemann
\[\zeta (m) = \sum^{\infty}_{n-1}\frac{1}{n^m}\]
\[\Rightarrow \varphi^{(m)}_{(1)}=(-1)^{m+1}\cdot m!\cdot \zeta (m+1),\ \ \ m=1,2,3...\]
\textbf{Expansión de Maclaurin}
\[
\begin{array}{rl}
	\ln\Gamma (z+1) =& \sum^{\infty}_{n=1}\frac{z^n}{n!}\varphi^{(n-1)}_{(1)}\\
	\\
	=& -\delta\cdot z +\sum^{\infty}_{n=2}(-1)^n\frac{z^n}{n}\zeta (n)
\end{array}
\]
converge en $|z|<1$ para $z=x\Rightarrow$ se puede calcular $\Gamma (z+1)$ para números reales o complejos.\\
\textbf{Suma de series}

Transformar la serie por medio de fracciones parciales y expresar la serie infinita como sumas finitas de funciones gamma y poligamma.\\
\textbf{\textcolor{red}{Ej.}}
\[
\begin{array}{rl}
	\varphi ^{(m)} (z+2)=& \varphi^{(m)} (z+1)+(-1)^{m}\frac{m!}{(z+1)^{m+1}}\\
	\varphi^{(m)} (z+2)-\varphi^{(m)} (z+1)=& \frac{(-1)^mm!}{(z+1)^{m+1}}\\
	\\
	\frac{d^{(m)}}{dz^{(m)}}\left[\varphi (z+2) -\varphi (z+1)\right]\\
	\\
	\varphi (z+2) - \varphi (z+1)=&\sum^{\infty}_{m=1}\frac{z+1}{m(m+z+1)}-\sum^{\infty}_{m=1}\frac{z}{m(m+z)}\\
	=& \sum^{\infty}_{m=1}\left[\frac{1}{m}-\frac{1}{m+z+1}-\frac{1}{m}+\frac{1}{m+z}\right]\\
	=& \frac{1}{z+1}
\end{array}
\]
\[\frac{d^m}{dz^m}\left[\varphi (z+2)-\varphi (z+1)\right] = \frac{d^m}{dz^m}\left[\frac{1}{z+1}\right]=\frac{(-1)^mm!}{(z+1)^{m+1}}\]
Demuestre que:
\[\frac{1}{2}\ln \left[\frac{\pi\cdot z}{sen(\pi\cdot z)}\right]=\sum^{\infty}_{n=1}\frac{\zeta (2n)}{2n}z^{2n},\ \ \ |z|<1\]
\textcolor{red}{
\begin{itemize}
	\item $\frac{1}{2}\ln\left[\textcolor{green}{\underbrace{\textcolor{red}{\Gamma (z)}}_{z\Gamma (z)=\Gamma (z+1)}}\Gamma (z-1)\right]=\frac{1}{2}\ln\left[\frac{\pi}{sen(z\pi)}\right]$
	\item $\ln \Gamma (z+1) =\lim\limits_{n\to \infty}\left[\ln (n!)+z\ln (n)-\ln (z+1)-...-\ln (z+n)\right]$
\end{itemize}
}
\[\Gamma (z)\Gamma (1-z)\rightsquigarrow\Gamma (1+z)\Gamma (1-z)\]

\[
\begin{array}{rl}
	\ln\left[\Gamma (1+z)\Gamma (1-z)\right]=& -\delta\cdot z +\sum\limits^{\infty}_{n=2}(-1)^n\frac{z^n}{n}\zeta (n)\\
	\\
	&+(-\delta(-z))+\sum\limits^{\infty}_{n=2}(-1)^n\frac{(-z)^n}{n}\zeta (n)\\
	\\
	=&\sum\limits^{\infty}_{n=2}\zeta (n)\underbrace{\left[\frac{z^n}{n}+\frac{(-z)^n}{n}\right]}_{\textcolor{red}{z^{2n}}}\\
	\\
	=& 2\sum\limits^{\infty}_{n=1}\cancelto{1}{(-1)^{2n}}\zeta (2n)\frac{z^{2n}}{2n}
\end{array}
\]
\newpage
\textbf{viernes 14 de julio}

\textbf{La Función Beta}

\[\Gamma (p+q)\propto \Gamma (p)\Gamma (q)\]
\[
\begin{array}{rlr}
	n!=& \Gamma (n+1)\& \int^{\infty}_0 e^{-t}t^2 dt,&\parbox{10em}{\footnotesize Variante de gamma para definir la funión beta} \\
	\\
	\Gamma (m)\Gamma (n)=& \int^{\infty}_0 e^{-u}u^m fu\int^{\infty}_0 e^{-v}v^n dv
\end{array}
\]
\[
\begin{array}{cc}
	u=x^2 & v=y^2\\
	du=2xdx & dv= 2ydy
\end{array}
\]
\[
\begin{array}{rl}
	\Gamma (m)\Gamma (n) =& \int^{\infty}_0 \int^{\infty}_0e^{-x^{2}}e^{-y^{2}}x^{2m}(2x)y^{2m}(2y)dxdy\\
	\\
	=& 4\int^{\infty}_0 e^{-x^{2}} e^{-x^{2}}e^{-y^{2}}x^{2m+1}y^{2m+1}dxdy
\end{array}
\]
otro cambio de variables
\[
\begin{array}{cc}
	x=rcos(\theta) & y=rsen(\theta)
\end{array}
\]
Usando el jacobiano
\[
\left|
\begin{array}{ccc}
	\frac{\partial x}{\partial r}= cos\theta dr && \frac{\partial x}{\partial \theta} =-rsen\theta d\theta\\
	\\
	\frac{\partial y}{\partial r} = sen\theta dr && \frac{\partial y}{\partial \theta} r cos\theta d\theta
\end{array}\right|
=rdrd\theta
\]
\[
\begin{array}{rl}
	=& \int^{\infty}_0\int^{\pi/2}_0 \cancelto{e^{-r^2cos^2\theta}}{e^{-r^2}} r^{2m+2n+3} cos^{2m+1}\theta sen ^{2n+1}drd\theta\\
	\\
	\Gamma (m)\Gamma (n) =& 2 \Gamma (m+n+1)\int^{\pi/2}_0 cos^{2m+1}\theta sen^{2n+1}\theta d\theta
\end{array}
\]
\[
\begin{array}{rl}
	\Gamma (z)=& 2 \int^{\infty}_0 e^{-t^2}t^{2z-1}dt;\ Re(z>0)\\
	\\
	\Gamma (q)\Gamma (p) =& 4\int^{\infty}_0 e^{-s^2} s^{2q-1}ds\int^{\infty}_0 e^{-t^2}t^{2p-1}dt
\end{array}
\]
\[
\begin{array}{cc}
	s=rcos\theta & t=rsen\theta\\
	r^2=s^2+t^2 & dsdt=rdrd\theta
\end{array}
\]
\[
\begin{array}{rl}
	\Gamma (q)\Gamma (p)=& 4\int\int e^{-r^2cos^2\theta}e^{-r^2sen^2\theta}r^{2q-1}cos^{2q-1}r^{2p-1}sen^{2p-1}rdrd\theta\\
	\\
	=& 2\cdot 2 \int^{\infty}_0 e^{-r^2}r^{2q+2p-1}cos^{2q-1}sen^{2p-1}drd\theta\\
	\\
	\textcolor{blue}{\Gamma (p)\Gamma (q)=}&\textcolor{blue}{2\int cos^{2q-1}sen^{2p-1} d\theta}
\end{array}
\]
\[\frac{m! n!}{(m+n+1)!}\textcolor{red}{\star}= 2\int^{pi /2}_0 cos^{2m+1}\theta sen^{2n+1}\theta d \theta\]

\begin{center}
\fbox{\begin{minipage}{15em}
  \[b(p,q)=\frac{\Gamma (q)\Gamma (p)}{\Gamma (q+p)}\]
\end{minipage}} $\Bigg\}$ definición de función beta
\end{center}
\[b(p,q)=b(q,p)\]
Si se sustituye $t=cos^2\theta$ y $dt=-2cos\theta sen\theta d\theta$
\[
\begin{array}{ll}
	\begin{array}{rl}
		\cancelto{t}{cos^2\theta}+sen^2\theta=&1\\
		sen^2\theta =&1-t
	\end{array} \textcolor{red}{\star}= & 2\int^{1}_0 cos^{2m}\theta\cancel{cos\theta} (1-t)^n\cancel{sen\theta}dt\\
	\\
	\begin{array}{rl}
		sen^{2n+1}\theta=&sen^{2n}\theta sen\theta\\
		=&(1-t)^n sen\theta
	\end{array} = & \int^{1}_0 t^m (1-t)^n dt\Rightarrow \text{ otra forma de escribir } B(m+1,n+1)
\end{array}
\]
\newpage
\textbf{lunes 17 de julio}

\textbf{\textcolor{red}{Ej.}}

Una particula de masa m se mueve en un potencial simétrico esta descrito por $v(x)=A|x|^n$ y energía total como $\frac{1}{2}m\left(\frac{dx}{dt}\right)^2 +v(x)=E$

Resolviendo para $\frac{dx}{dt}$ e intgrando encontramos que el movimiento períodico es:
\[\zeta=\sqrt{2m}\int^{X_{max}}_0 \frac{dx}{(E-Ax)^{1/2}}\]
donde $X_{max}$ es el punto de innplexion clasico dado por $Ax_{max}^n=E$

Muestre que:
\[\uptau=\frac{2}{n}\sqrt{\frac{e\pi m}{E}}\left(\frac{E}{A}\right)^{1/2}\left[\frac{\Gamma (1/n)}{\Gamma (\frac{1}{n}+\frac{1}{2})}\right]\]
\begin{center}
\fbox{
\begin{minipage}{30em}
  \[B(p+1,q+1)=B(1/n,1/2)=\int^1_0t^p(1-t^qdt)\]
  \[\textcolor{green}{\frac{1}{n}=p+1\Rightarrow p=\frac{1}{n}-1}\]
  \[\textcolor{green}{\frac{1}{2}=q+1\Rightarrow q=\frac{1}{2}-2}\]
  \[\textcolor{green}{\int^1_0t^{\frac{1}{n}-1}(1-t)^{-1/2}}\]
\end{minipage}}
\end{center}
\[\textcolor{red}{t=\frac{Ax^n}{E}}\Rightarrow \left(\frac{Et}{A}\right)^{1/n}\]
\[\textcolor{red}{dt=\frac{A}{E}n(x^{n-1})dx}\]
\[\textcolor{red}{dx=\underbrace{\frac{E}{A}}_{\textcolor{black}{\frac{1}{n}\frac{x}{t}dt}}\frac{1}{n}\frac{x}{x^n}dt}\]
\[
\begin{array}{rl}
	\uptau =& 2\frac{\sqrt{2m}}{\sqrt{E}\int^1_0\frac{\frac{x}{nt}dt}{(1-t)^{1/2}}}\\
	=&\textcolor{red}{\left(\frac{E}{A}\right)^{1/n}\frac{2}{n}}\sqrt{\frac{2m}{E}}\int^1_0\frac{\frac{\frac{(Et)^{1/n}}{A^{1/n}}}{nt}dt}{(1-t)^{1/2}}\\
	=&\frac{2}{n}\sqrt{\frac{2m}{E}}\left(\frac{E}{A}\right)^{1/n}\underbrace{\int^1_0t^{\frac{1}{n}-1}(1-t)^{-1/2}dt}_{\textcolor{red}{\frac{\Gamma (\frac{1}{n}) \cancelto{\sqrt{\pi}}{\Gamma (\frac{1}{2})}}{\Gamma (\frac{1}{n}+\frac{1}{2})}}}\\
	\uptau =& \frac{2}{n}\sqrt{\frac{2\pi m}{E}}\left(\frac{E}{A}\right)^{1/n}\frac{\Gamma (\frac{1}{n})}{\Gamma (\frac{1}{n}+\frac{1}{2})}
\end{array}
\]
\section{SERIES DE STIRLING}
\[\ln (n!)\rightsquigarrow \ln \Gamma (z)\]
para z no entero, $|z|$ muy grande, no Series de Maclaurin
\[\Gamma (z+1)=z\Gamma (z)\]
Se usa una expansión asintótica para $\ln (\Gamma (z))$ esta expansión es la que se conoce como \textcolor{red}{Serie de Stirling} o \textcolor{red}{Formula de Stirling}

\textbf{Series asintóticas$\to$ Series semi-convergentes$\to$Series de Poincaré}

Se consideran 2 tipos de integrales 
\[
\begin{array}{rl}
	I_1 (x)=& \int^{\infty}_xe^{-u}f(u)du\\
	\\
	I_2 (x)=& \int^{\infty}_0e^{-u}f\left(\frac{u}{x}\right)du
\end{array}
\]
Qué podemos hacer con las series asintóticas:
\begin{enumerate}
	\item Si multiplicamos 2 series asintóticas vamos a tener otra serie asintótica.
	\item Se pueden integrar término a término u el resultado será, otra serie asintotica de la forma
	\[\int^{\infty}_xf(x)dx\]
	\item La diferenciación es válido únicamente bajo ciertas condiciones.
\end{enumerate}

Método para generar una serie asintótica:
\[\begin{array}{ll}
	\textbf{Método de Steepest Descents}
	&\left\}\begin{array}{ll}
		\text{a lo largo del camino}\\
		\text{máximo decrecimiento}	
	\end{array}\right.\\
	\to \text{Saddle point}\to \text{punto silla}\\
	\to \text{Método de integrandos Holomorfos}
\end{array}\]
\textbf{Derivación de la fórmula de integración de Euler-Maclaurin}
\[\longrightarrow \text{uno de los usos de los polinomios de Bernoulli}\longleftarrow\]
Números de Bernoulli $\longrightarrow$ Polinomios de Bernoulli
\[B_n\]
\[\frac{t}{e^t-1}=\sum^{\infty}_{n=0}\frac{B_nt^n}{n!}\]
Como $\frac{t}{e^t-1}=\sum^{\infty}_{n=0}\frac{B_nt^n}{n!}$ es una serie de Taylor se pueden identificar a $B_n$ como una sucesión de derivadas de la función generadora valuadas en cero.
\[\Rightarrow B_n=\frac{d^n}{dt^n}\left.\left(\frac{t}{e^t-1}\right)\right|_{t=0}\]
\[
\begin{array}{rl}
	B_0=&\lim\limits_{n\to 0}\frac{t}{e^t-1}=1\\
	\\
	B_1=&\lim\limits_{n\to 0}\frac{d}{dt}\left.\left[\frac{t}{e^t-1}\right]\right|_{t=0}=-\frac{1}{2}\\
	\\
	B_1=& \lim\limits_{n\to 0}\left(\frac{1}{e^t-1}-\frac{te^t}{(e^t-1)^2}\right)\\
	\\
	B_1=& \lim\limits_{n\to 0}\left(\frac{e^t-1-te^t}{(e^t-1)^2}\right)=\lim\limits_{n\to 0}\frac{\cancel{e^t}-t\cancel{e^t}-\cancel{e^t}}{2(e^t-1)\cancel{e^t}}\\
	\\
	=& -\frac{1}{2}
\end{array}\]
\textbf{La fórmula Euler-Maclaurin para evaluar integrales definidas en $(0,\infty)$}
\[
\begin{array}{rl}
	\int^{\infty}_0=&\frac{1}{2}f(0)+f(1)+f(2)+f(3...)\\
	\\
	&+\frac{B_2}{2!}f'(0)+\frac{B_4}{4!}f^{(3)}(0)+\frac{B_6}{6!}f^{(5)}(0)
\end{array}\]
Donde $B_n$ son los números de Bernoulli
\[B_2=\frac{1}{6},\ B_4=-\frac{1}{30},\ B_6=\frac{1}{42},\ B_8=-\frac{1}{30},...\]
\[\int^{\infty}_0\frac{dx}{(z+x)^2}=\frac{1}{z}\]
\begin{center}
	z no está en el eje real negativo
\end{center}
\[f(1)+f(2)+f(3)+...=\sum^{\infty}_{n=1}\frac{1}{(z+n)^2}=\]
\[
\begin{array}{rl}
	\textcolor{red}{\varphi^{(m)}(z+1)}=&\textcolor{red}{\frac{d^{m+1}}{dz^{m+1}}\ln \Gamma (z+1)}\\
	\\
	\underset{m=1,2,3...}{ }=& (-1)^{m+1}m!\sum^{\infty}_{n=2}\frac{1}{(z+n)^{m+1}}
\end{array}
\]
\textbf{m=1}
\[(-1)^2\cdot 1!\sum^{\infty}_{n=1}\frac{1}{(z+n)^2}=\varphi^{(1)}(z+1)\]
\[f^{(2n+1)}_{(0)}=\left.\left(\frac{d}{dx}\right)^{2n-1}\frac{1}{(z+x)^2}\right|_{x=0}=-\frac{(2n)!}{z^{2n+1}}\]
\[
\begin{array}{rl}
	\int^{\infty}_0f(x)dx=\int^{\infty}_0\frac{dx}{(z+x)^2}=&\frac{1}{z}=\frac{1}{2}\frac{1}{z^2}+\underbrace{\varphi ^{(1)}_{(z+1)}}_{\textcolor{red}{\text{poligamma con m=1}}}-\frac{B_2}{z^3}-\frac{B_4}{z^5}-...\\
	\varphi ^{(1)}_{(z+1)}=& \frac{d}{dz}\underbrace{\varphi (z+1)}_{digamma}\\
	\varphi ^{(1)}_{(z+1)}=&\frac{1}{z}-\frac{1}{2z^2}+\frac{B_2}{z^3}+\frac{B_4}{z^5}+...\\
	=& \frac{1}{z}-\frac{1}{2z^2}+\underbrace{\sum^{\infty}_{n=1}\frac{B_{2n}}{z^{2n+1}}}_{\parbox{5.5em}{\tiny Es una serie asintótica que es util si se utilizan pocos términos}}
\end{array}\]
Al integrar $\varphi ^{(1)}_{(z+1)}$ obtenemos $\varphi _{(z+1)}$ [Digamma]
\[
\begin{array}{rl}
	\varphi_{(z+1)}=& C_1+\ln z+\frac{1}{2z}-\frac{B_2}{2z^2}-\frac{B_4}{4z^4}-...\\
	\\
	=& C_1+\ln z+\frac{1}{2z}-\sum\limits^{\infty}_{n=1}\frac{B_{2n}}{\cdot z^{2n}}
\end{array}
\]
\newpage
\textbf{viernes 21 de julio}
\textbf{Fórmula de stirling}
\[
\begin{array}{rl}
	\varphi_{(z+1)}=& C_1 +\ln z+ \frac{1}{2z}-\frac{B_2}{2z^2}-\frac{B_4}{4z^4}-...,\ \ \ B_{2n} \text{ Números de Bernoulli}\\
	\\
	=& C_1 +\ln z+ \frac{1}{2z}-\sum\limits^{\infty}_{n=1}\frac{B_{2n}}{2nz^{2n}}
\end{array}
\]
\[\textcolor{red}{\varphi_(z+1)\equiv \frac{d}{dz} \ln \Gamma (z+1)}\]
\[
\begin{array}{rl}
	\ln \Gamma (z+1) =& C_2 +\left(z+\frac{1}{2}\right) \ln z + (c_1-1)z+\frac{B_2}{2z}+...+ \frac{B_{2n}}{2n(2n-1)z^{2n-1}}+...\\
	\\
	&\Rightarrow C_1=0\ y \ c_2=\frac{1}{2}\ln 2\pi\\
	\\
	\ln \Gamma (z+1) =&\frac{1}{2} \ln 2\pi + \left(z+\frac{1}{2}\right) \ln z -z+\frac{1}{12z}+\frac{1}{360z^3}+\frac{1}{1260z^5}+...\\
\end{array}
\]
\[
\begin{array}{rl}
	\lim\limits_{n\to \infty}\frac{z!}{\sqrt{z}z^ze^{-z}}=& \sqrt{2\pi}\\
	\\
	z!\approx &\sqrt{2\pi}\sqrt{z}z^ze^{-z}\\
	\\
	\textcolor{blue}{\Gamma (z+1)\approx} & \textcolor{blue}{\sqrt{2\pi} z^{z+\frac{1}{2}}e^{-z}}\\
	\\
	\ln \Gamma (z+1) \approx & \frac{1}{2} \ln 2\pi +\left( z+\frac{1}{2}\right)(\ln z)-z\\
	\\
	\ln n! \approx & n\ln n - n 
\end{array}
\]
\textbf{\textcolor{red}{Ej.}}

n es natural.
\[
\begin{array}{lcr}
	\lim\limits_{n\to \infty} \sqrt[n]{n}=1 && \lim\limits_{n\to \infty} \sqrt[n^3]{n}=1\\
	\\
	&\lim\limits_{n\to \infty} \sqrt[n^3]{n^7}=1\\
	\\
	\lim\limits_{n\to \infty} \sqrt[p(n)]{q(n)}=1\\
	\\
	\text{pero...}\\
	\\
	\lim\limits_{n\to \infty} \sqrt[n]{n!}=\infty
\end{array}
\]
\[
\begin{array}{rl}
	\Rightarrow \lim\limits_{n\to \infty} \sqrt[n]{n!}=& \lim\limits_{n\to \infty}\sqrt[n]{\sqrt{2\pi}\sqrt{n}n^ne^{-n}}\\
	\\
	=& \lim\limits_{n\to \infty} \cancelto{1}{\sqrt[2n]{2\pi n}}\cdot n\cdot \cancelto{\frac{1}{e}}{e^{-1}}\\
	\\
	=& \infty
\end{array}
\]
\[\lim\limits_{n\to \infty} \frac{2n!}{4^n(n+1)!n!}\textcolor{red}{=0}\]
\[\textcolor{red}{
\begin{array}{rl}
	(n+1)!=& (n+1)n!(\sqrt{2\pi n } n^n e^{-n})\\
	\\
	2n!=& \sqrt{2\pi}\sqrt{2n}(2n)^{2n}e^{-2n}
\end{array}
}\]
\[\lim\limits_{n\to \infty}\frac{\sqrt{2\pi} \sqrt{2n}\textcolor{red}{\cancel{\textcolor{black}{(2n)^{2n}}}\cancel{\textcolor{black}{(e)^{-2n}}}}}{\textcolor{red}{\cancel{\textcolor{black}{4^n}}}(n+1)(\sqrt{2\pi n}\cdot \cancelto{2\pi n}{n^n}\cancel{e^{-n}})^2}\]
\[\lim\limits_{n\to \infty} \frac{\sqrt{\pi n}}{(n+1)(\pi n)}=0\]
\textbf{\textcolor{red}{Ej.}}

Un conjunto de N partículas distiguibles es asignado a los estados $\psi_i\ni i=1,2,...,M$. Si el número de partículas en los diversos estados son $n_1,n_2,n_3,...,n_M$ (con $M\ll N$). El número de formas que se pueden hacer es:
\[w=\frac{N!}{n_1!n_2!...n_M!}\]
La entropía asociada es $S=k\ \ln w$ donde k es la constante de Boltzmann.\\
En el límite cuando $N\to \infty$, con $n_i=P_i N$, ($P_i$ es la fracción del as partículas en le estado i). Encuentre S  en función de $N$ y $P_i$.
\[\textcolor{red}{\text{Pista: } \sum_iP_i=1}\] 
\textbf{Sol. }
\[
\begin{array}{rl}
	\ln N!\approx & N\ln N-N\ \ \text{usando la condición} \sum_in_i=N\\
	S=k\ln W=& l \left[\ln N!- \ln n_i!\right]\\
	\\
	=& k\left[ (N\ln N-N)-\left(\sum^M_{i=1}n_i+nn_i-n_i\right)\right]
\end{array}
\]


\newpage
\textbf{lunes 24 de julio}

\[
\begin{array}{rlr}
	\zeta (z) =& \sum^{\infty}_{n=1} n^{-z} & \text{por definición}\\
	\zeta (z) =& \int^{\infty}_{0}\frac{u^{z-1}}{e^u-1}du
\end{array}
\]
\[
\begin{array}{lcl}
	\zeta (1)=\infty && \zeta (2) =\frac{\pi^2}{6}=\sum^{\infty}_{n=1}\frac{1}{n^2}\\
	\zeta (3) = 1.202056 && \zeta (4)=\frac{\pi^4}{90}\\
	\zeta (5) = 1.0369 &&  \zeta (6) = \frac{\pi^6}{945}\\
	&Re(z)>1
\end{array}
\]
\hfill
\[
\begin{array}{rl}
	\zeta (-1)=& \frac{1}{1^{-1}}+\frac{1}{2^{-1}}+...\\
	=& 1+2+3+4+...\\
	=&-\frac{1}{12}
\end{array}
\]
\begin{center}
	$\zeta(-2n)=0$\\
	Ceros "triviales"
\end{center}

\[
\begin{array}{lccc}
	\begin{array}{ll}
		z=2+i\\
		n=\frac{1}{2}
	\end{array} & n^z=&\frac{1}{2}^{2+i}=& \frac{1}{2}^2\frac{1}{2}^i
\end{array}
\]
Otra forma de definir Zeta de Riemann
\[\zeta (s)=\prod^{\infty}_{\parbox{2em}{\footnotesize primos p= 2}}(1-p^{-s})^{-1}\]
franja crítica:
\[
\begin{array}{rll}
	\zeta (z)=& \frac{1}{\Gamma (z)}\int^{\infty}_0\frac{t^{z-1}}{e^t-1}dt\frac{e^{-t}}{e^{-t}}=\frac{1}{\Gamma (z)}\int^{\infty}_0\frac{t^{z-1}\cdot e^{-t}}{1-e^{-t}}\\
	\\
	=& \int^{\infty}_0 t^{z-1}e^{-t}\sum^{\infty}_{m=1}e^{-mt}dt & \textcolor{blue}{\left(\frac{1}{1-x}=\sum^{\infty}_{m=0}x^m\right)}\\
	\\
	=& \frac{1}{\Gamma (z)} \int^{\infty}_0 t^{z-1} \sum^{\infty}_{m=1} e^{-mt}dt &=\frac{1}{\Gamma (z)}\int^{\infty}_0\sum^{\infty}_{m=1}\left(\frac{x}{m}\right)^{z-1}e^{-x}\frac{dx}{m}\\ 
	\\
	&&=\frac{1}{\Gamma (z)}\left(\sum^{\infty}_{m=1} \frac{1}{m^z}\right)\int^{\infty}_0x^{z-1}e^{-x}dx
\end{array}
\]
\textbf{\textcolor{red}{Ej.}}
\[
\begin{array}{rlcc}
	\int^{\infty}_0 \frac{x^n\cdot e^x}{(e^x-1)^2}dx=&n!\zeta (n) &\textcolor{blue}{\begin{array}{cc}
		u=x^n\\
		du=nx^{n-1}dx
	\end{array}}	& \textcolor{blue}{\begin{array}{cc}
		v=-\frac{1}{e^x-1}\\
		dv=\frac{e^x}{(e^x-1)^2}dx
	\end{array}}\\
	\\
	=&-\cancelto{0}{\left.\frac{x^n}{e^x-1}\right|^{\infty}_0}-\int^{\infty}_0-\frac{1}{e^x-1}nx^{n-1}dx&& \zeta (z)=\frac{1}{\Gamma (z)}\int^{\infty}_0 \frac{t^{z-1}}{e^t-1}dt\\
	\\
	=&\int^{\infty}_0 \frac{x^{n-1}}{e^x-1}dx=n(n+1)!\zeta (z)
\end{array}
\]
\section*{Funciones Incompletas}
\textbf{\underline{Gamma incompleta}}
\[
\begin{array}{rlr}
	\delta (a,x)=& \int^x_0e^{-t}t^{a-1}dt& R(a)>0\\
	\\
	\Gamma (a,x)=&\int^{\infty}_x e^{-t}t^{a-1}dt\\
	\\
	\delta (a,x)+ \Gamma (a,x)=& Gamma(a)=\Gamma (a)
\end{array}
\]

Cuando a es entero:
\[
\begin{array}{rl}
	\delta (n,x)=& (n-1)!\left(1-e^{-x}\sum^{n-1}_{s=0}\frac{x^s}{s!}\right)\\
	\\
	\Gamma (n,x)=&(n-1)!e^{-x}\sum^{n-1}_{s=0}\frac{x^s}{s!}
\end{array}
\]
Cuando a no es entero:
\[
\begin{array}{rll}
	\delta (a,x)\rightarrow & \text{serie de potencias para x pequeno} &\Rightarrow \delta (a,x)=x^a\sum^{\infty}_{n=0}(-1)^n\frac{x^n}{n!(a+n)}\\
	\Gamma (a,x)\rightarrow & \text{serie asintótica} &\Rightarrow \Gamma (a,x)\sim x^{a-1}e^{-x}\sum^{\infty}_{n=0}\frac{\Gamma (a)}{\Gamma (a-n)}\cdot\frac{1}{x^n}
\end{array}
\]
Cuando a=0
\[\Gamma (0,x)\sim \frac{e^x}{x}\sum^{\infty}_{n=0}(-1)^n \frac{n!}{x^n}\]
\textbf{\underline{Beta Incompleta}}
\[
\begin{array}{rlr}
	B_x (p,q)=&\int^{x}_0t^{p-1}(1-t)^{z-1}dt
	& 0\leq x\leq 1,p>0,q>0
\end{array}
\]
\[Si\ \ x=0\Rightarrow B(p,q) \text{  función Beta}\]
\textcolor{blue}{Aparece en:
\begin{itemize}
	\item Funciones hipergeométricas
	\item Probabilidades
\end{itemize}}

\newpage
\textbf{Viernes 28 de julio}

\newpage
\textbf{Lunes 31 de julio}
\section*{Derivaci'on bajo el signo de la integral}
\textbf{Integral de Feynmann}

\[\frac{d}{dt}\int f(x,t)dx=\int\frac{\partial (x,t)}{\partial t}dx\]
\textbf{Derivaci'on de la funci'on Gamma}
\[F(t)=\int\limits^{\infty}_0e^{-tx}dx\]
$F(t)=\frac{1}{t}$ para $t>0$
\[
\begin{array}{rl}
	\frac{dF(t)}{dt}=F'(t)=& \int\limits^{\infty}_0\frac{d}{dt}e^{-tx}dx=\frac{d}{dt}\left[\frac{1}{t}\right]\\
	\Rightarrow & \int\limits^{\infty}_0-xe^{-tx}dx=-\frac{1}{t^2}\\
	\Rightarrow & \int\limits^{\infty}_0x^ne^{-tx}dx=\frac{n!}{t^{n+1}}
\end{array}
\]
$\textcolor{red}{n!=\int\limits^{\infty}_0e^{-x}x^ndx}$
\[n!=t^{n+1}\int\limits^{\infty}_0x^ne^{-tx}dx=\int\limits^{\infty}_0x^ne^{-x}dx\]
\[\begin{array}{lr}
	\textcolor{red}{\begin{array}{l}
		u=tx\\
		du=tdx
	\end{array}} & \int\limits^{\infty}_0\cancel{t^n}\cancel{t}\frac{u^n}{\cancel{t^n}}e^{-u}\frac{du}{\cancel{t}}
\end{array}\]
\par\noindent\rule{\textwidth}{0.5pt}
\[\int\limits^{1}_0\frac{x^2-1}{\ln (x)}\]
\[
\begin{array}{rl}
	\textcolor{red}{\frac{d}{dt} I(t)=}&\textcolor{red}{\int\limits^{1}_0\frac{\partial}{\partial t}\frac{x^t-1}{\ln (x)}}\\
	=&\int\limits^{1}_0\frac{1}{\ln (x)}\frac{\partial}{\partial t}[x^t-1]dx\\
	=&\int\limits^{1}_0\frac{1}{\cancel{\ln (x)}}x^t\cancel{\ln (x)}dx\\
	=&\int\limits^{1}_0x^tdx=\left.\frac{1}{t+1}x^{t+1}\right|^1_0\\
	=&\frac{1}{t+1}\\
	\int\frac{d}{\cancel{dt}}I(t)=& \int \frac{1}{t+1}dt\\
	I(t)=& \ln |t+1|+c
\end{array}
\]
si t=0
\[\begin{array}{rl}
	I(2)=& \ln |2+1|=\ln |3|\\
	\textcolor{red}{I(2)=}&\textcolor{red}{\int\limits^2_0 I'(t)dt}\\
	=& \int\limits^2_0 \frac{1}{t+1} dt=\left.\vphantom{\frac{0}{0}}\ln |t+1|\right|^2_0\\
	= & \ln |3|
\end{array}\]
\par\noindent\rule{\textwidth}{0.5pt}
\[\mathbf{\int\limits^{\infty}_0\frac{e^{-x^2}sen(x^2)}{x^2}}dx\]
\[\textcolor{purple}{\frac{d}{dx}sen(ax^2)=x^2cos(ax^2)}\]
\[
\begin{array}{rl}
	\textcolor{purple}{\frac{d}{da}}I(a)=& \cancel{\frac{d}{da}}\int\limits^{\infty}_0\textcolor{purple}{\frac{\partial}{\partial a}}\frac{e^{-x^2}sen(x^2)}{x^2}dx\\ \\
	=& \int\limits^{\infty}_0\frac{e^{-x^2}}{\cancel{x^2}}\cancel{x^2}cos(ax^2)dx\\ \\
	I'(a)=&\int\limits^{\infty}_0e^{-x^2}cos(ax^2)dx\\
	=& Re\int\limits^{\infty}_0e^{-x^2}e^{iax^2}dx\\ \\
	=& Re\int\limits^{\infty}_0e^{-x^2(1-ia)}dx\\ \\
	\textcolor{red}{\int}I'(a)\textcolor{red}{da}=&\textcolor{red}{\int}\frac{1}{2}\sqrt{\pi} Re\left[\frac{1}{\sqrt{1-ia}}\right]\textcolor{red}{da} \\ \\
	
	I(a)=& \frac{\sqrt{\pi}}{\cancel{2}}Re\left[\cancel{2}i\sqrt{1-ia}\right]+c
\end{array}
\]
\[a=0\Rightarrow sen(ax^2)=0\Rightarrow c=0\]
a=1
\[I(1)=\sqrt{\pi}Re\left[i\sqrt{1-i}\right]\]
$1-i$ en forma polar: $re^{i\theta}$
\[
\begin{array}{rl}
	r=& \sqrt{1^2+(-1)^2}=\sqrt{2}\\ \\
	\theta=& tan^{-1}\left[\frac{-1}{1}\right]=-\frac{\pi}{4}\\ \\
	1-i=& \sqrt{2}e^{-\frac{\pi}{4}i}\\ \\
	\textcolor{purple}{\sqrt{1-i}=}&\textcolor{purple}{\sqrt{\sqrt{2}}e^{-\frac{\pi}{8}i}}\\ \\
	 \textcolor{purple}{=}& \textcolor{purple}{\sqrt{\sqrt{2}}\left[cos\frac{\pi}{8}-i sen\frac{\pi}{8}\right]}\\ \\
	 \textcolor{purple}{i\sqrt{1-i}=}&\textcolor{purple}{\sqrt{\sqrt{2}}\left[icos\frac{\pi}{8}-i^2 sen\frac{\pi}{8}\right]}\\ \\
\end{array}
\]
\par\noindent\rule{\textwidth}{0.5pt}
\[\textcolor{blue}{\mathbf{\int\limits^{\infty}_0\left[\vphantom{\frac{0}{0}}\ln (x)\right]\left[\vphantom{\frac{0}{0}}\ln (tanh(x))\right]dx}}\]
\[I(m)=\int\limits^{\infty}_0x^m\ln(tanh(x))dx\]
\begin{center}
	a usar:
\end{center}
\[\frac{d}{dm}x^m=x^m\ln (x)\]
\[tanh(x)=\frac{e^x-e^{-x}}{e^x+e^{-x}}=\frac{1-e^{-2x}}{1+e^{-2x}}\]
\[\Gamma (z)=\int\limits^{\infty}_0t^{z-1}e^{-t}dt\]
\newpage
\textbf{Viernes 04 de agosto}
\textbf{\textcolor{red}{Ej.}}
\[	\int\limits^{\infty}_0\left[\ln (x)\right]\left[\ln (tanh(x))\right]\]
\[
\begin{array}{rl}
	I(m)=& \int\limits^{\infty}_0 x^m\left[ \ln (1-e^{-2x}-\ln (1+e^{-2x})\right]dx\\ \\
	=& \int\limits^{\infty}_0 x^m\left[ \sum\limits^{\infty}_{n=1}\frac{(-1)^{n+1}(-e)^{-2nx}}{n} \right]dx -\int\limits^{\infty}_0 x^m \sum\limits^{\infty}_{n=1} \frac{(-1)^{n+1}(e)^{-2nx}}{n}dx\\ \\
	\\
  


\end{array}
\]
\begin{center}
\fbox{\begin{minipage}{12cm}
\begin{align*}
	tanh(x)=&\frac{e^x-e^x}{e^x+e^{-x}}\\
	\frac{d}{dm}x^m =& x^m log(x)\\
	tanh(x)=& \frac{1-e^{-2x}}{1+e^{-2x}}\\
	\text{Serie de MacLaurin}\\
	\ln (1+x)=& \sum\limits^{\infty}_{n=1}(-1)^{n-1} \frac{x^n}{n}
\end{align*}
\end{minipage}}
PEDIR COPIA
\section*{Funciones Bessel}
\begin{align*}
	x^2y''+xy'+(x^2-p^2)y=0 && \text{EC. Bessel de orden P.}\\
	x^2y''+xy'+(x^2-\frac{1}{4})y=0 && \text{EC. Bessel de orden } \frac{1}{2}\\
\end{align*}
\end{center}

\textbf{M'etodo por series de Frobenius}
\[y=\sum\limits^{\infty}_{n=0}c_nx^{n+r}\]
\begin{align*}
	xy''+y'+xy=0 && \text{Ec. Bessel de orden 0}
\end{align*}
\[y_2=y_1\ln (x) +x^{r_1+1}\sum\limits^{\infty}_{n=0} b_nx^n\]
\par\noindent\rule{\textwidth}{0.5pt}
\[x\sum\limits^{\infty}_{n=0} (n+r)(n+r-1)C_nx^{\overset{\textcolor{red}{n+r-1}}{n+r-2}}+\sum\limits^{\infty}_{n=0}(n+r)C_nx^{n+r-1}+\sum\limits^{\infty}_{n=0}C_nx^{\overset{\textcolor{red}{n+r-1}}{n+r}}=0\]
\begin{align*}
	\textcolor{blue}{n=0} && \textcolor{blue}{(r)(r-1)C_0x^{r-1}} && \textcolor{blue}{rC_0x^{r-1}}&&\\
	n=1 && (1+r)(r)C_1x^r && (1+r)C_1x^r\\
	&& \sum\limits^{\infty}_{n=0} (n+r)(n+r-1)C_nx^{n+r-1} && \textcolor{blue}{\sum\limits^{\infty}_{n=0} (n+r) C_nx^{n+r-1}}
\end{align*}

\newpage
\textbf{lunes 07 d e agosto}

\[\boxed{xy''+y'+xy=0}\]
\[
\begin{array}{rl}
	y_1=& C_0\sum\limits^{\infty}_{n=0} \frac{(-1)^nx^{2n}}{2^{2n}(n!)^2}\\\\
	J_0(x)=&\sum\limits^{\infty}_{n=0}\frac{(-1)^nx^{2n}}{2^{2n}(n!)^2}\\ \\
	y_2=& y_1\ln (x)+x^{r_2+1}\sum\limits^{\infty}_{n=1}b_nx^n\\
	&\boxed{xy''\ln (x)} +2y_1'-\cancel{\frac{1}{x}y_1}+\sum\limits^{\infty}_{n=1}n(n-1)b_nx^{n-1}+\\
	&+\boxed{y_1'\ln (x)} +\cancel{\frac{1}{x}y_1}+\sum\limits^{\infty}_{n=1}nb_nx^{n-1}+\\
	&+\boxed{xy_1\ln (x)} +\sum\limits^{\infty}_{n=1}b_nx^{n+1}=0 \\\\
	=&2\sum\limits^{\infty}_{n=1}\frac{(-1)^n(2n)x^{2n-1}}{2^{2n}(n!)^2}+\sum\limits^{\infty}_{n=1}n(n-1)b_nx^{\mathcircled{n-1}}+\sum\limits^{\infty}_{n=1}nb_nx^{\mathcircled{n-1}}+\sum\limits^{\infty}_{n=1}b_nx^{\mathcircled{n-1}}=0
\end{array}\]
\begin{align*}
	\underline{x^0}:&& n=1 && 1(0)b_1+1b_1=0&&\Rightarrow b_1=0\\
	\underline{x^1}:&& &&\frac{2(-1)(2)}{2^2(1!)^2}+2(2-1)b_2+2b_2=0&&\Rightarrow -1+2b_2+2b_2=0\\
	&& && && \Rightarrow b_2=\frac{1}{4}\\
	\underline{x^2}:&& && \text{Ya funciona para todas las series}
\end{align*}
\par\noindent\rule{\textwidth}{0.5pt}
\[2\sum\limits^{\infty}_{n=2}\frac{(-1)^n(2n)x^{2n-1}}{2^{2n}(n!)^2}+\sum\limits^{\infty}_{n=3}n(n-1)b_nx^{n-1}+\sum\limits^{\infty}_{n=3}nb_jx^{n-1}+\sum\limits^{\infty}_{n=1}b_nx^{n+1}=0\]
\[2\sum\limits^{\infty}_{n=2}\frac{(-1)^n(2n)x^{2n-1}}{2^{2n}(n!)^2}+\sum\limits^{\infty}_{n=2}(n+1)(n)b_{n+1}x^n+\sum\limits^{\infty}_{n=2}(n+1)b_{n+1}x^n+\sum\limits^{\infty}_{n=2}b_{n-1}x^n=0\]
\[2\sum\limits^{\infty}_{n=2}\frac{(-1)^n(2n)x^{2n-1}}{2^{2n}(n!)^2}+\sum\limits^{\infty}_{n=2}\underset{\textcolor{red}{(n+1)^2b_{n+1}\cancel{(n+1)}+b_{n-1}}}{\left[n(n+1)b_{n+1}+(n+1)b_{n+1}+b_{n-1}\right]}x^n=0\]
\[2\sum\limits^{\infty}_{n=2}\frac{(-1)^n(2n)x^{2n-1}}{2^{2n}(n!)^2}+\sum\limits^{\infty}_{n=2}\left[(n+1)^2b_{n+1}+b_{n-1}\right]x^n=0\]
\[\boxed{\text{n par}\rightarrow \text{n=2k}}\]
\begin{align*}
	(2k+1)^2b_{2k+1}+b_{2k-1}&& \text{para }k\geq 1
\end{align*}
\[b_{2k+1}=\frac{-b_{2k-1}}{(2k+1)^2}\]
\textbf{\underline{k=1}:}
\[b_3=\frac{-b_1}{3^2}=0\]
\textbf{\underline{k=2}:}
\[b_5=\frac{-b_3}{5^2}=0\]
\[\boxed{\text{n impar}\rightarrow \text{n=2k-1}}, k\geq 2\]
\[\frac{2(-1)^{k}2k}{2^{2k}((k)!)^2}+(2k)^2b_{2k}+b_{2k-2}=0\]
\[(2k)^2b_{2k}+b_{2k-2}=\frac{-(-1)^k4k}{2^{2k}(k!)^2}\rightsquigarrow \left[\underset{\textcolor{red}{\star}}{(2k)^2\frac{(-1)^{k+1}C_{2k}}{\textcolor{blue}{\cancel{\textcolor{black}{2^{2k}(k!)^2}}}}}+\underbrace{\underset{\textcolor{red}{\star\star}}{\frac{(-1)^kC_{2k-2}}{2^{2k-2}((k-1)!)^2}}}_{\textcolor{blue}{\frac{2^2k(k!)^2}{2^{2k-2}((k-1)!)^2}=4k^2}}=-\frac{(-1)^k4K}{2^{2k}(k!)^2}\right]\textcolor{blue}{2^2k(k!)^2}\]
suponemos:
\[\textcolor{red}{\star} b_{2k}=\frac{(-1)^{k+1}C_{2k}}{2^{2k}(k!)^2}\]
\[\underline{k=1}: b_2=\frac{(-1)^2C_2}{2^2(1!)^2}=\frac{C_2}{2^2}\Rightarrow C_2=1\]
\[\textcolor{red}{\star\star}b_{2k-2}=\frac{(-1)^kC_{2k-2}}{2^{2k-2}((k-1)!)^2}\]
entonces:
\[4k^2(-1)^k(-1)C_{2k}+(-1)^kC_{2k-2}4k^2=-(-1)^k4k\]
\[c_{2k-2}-C_{2k}=\frac{-1}{k}\]
\[C_{2k}=C_{2k-2}+\frac{1}{k}\]
Buscando el patron:\\
\underline{k=2}:
\[C_4=C_2+\frac{1}{2}=1+\frac{1}{2}=H_2\]
\underline{k=3}:
\[C_6=C_4+\frac{1}{3}=\left[1+\frac{1}{2}\right]+\frac{1}{3}=H_3\]
\underline{k=4}:
\[C_8=C_6+\frac{1}{4}=1+\frac{1}{2}+\frac{1}{3}+\frac{1}{4}=H_4\]
\[C_{2k}=H_k\]
\begin{align*}
	b_{2k}=&\frac{(-1)^{k+1}C_{2k}}{2^{2k}(k!)^2}\\
	=&\frac{(-1)^{k+1}H_k}{2^{2k}(k!)^2}
\end{align*}
\begin{align*}
	y_2=& J_0(x)\ln (x)+\sum\limits^{\infty}_{n=1}\frac{(-1)^{n+1}H_nx^{2n-1}}{2^{2n}(n!)^2}\\
	Y_o(x)=&\frac{2}{\pi}(\gamma -\ln (2))y_1+\frac{2}{\pi}y_2
\end{align*}
\[\gamma=\text{constante Euler-Mascheroni}\]
\[\gamma=\lim\limits_{n\to \infty}(H_n-\ln (n))=0.577215\]
\textbf{La soluci'on general}
\[y=C_1J_0(x)+C_2Y_0(x)\]
\par\noindent\rule{\textwidth}{0.5pt}
\[\underset{\langle\text{EC. de Bessel de orden P}\rangle}{x^2y''+xy'+(x^2-p^2)y=0}\]
\[y=\sum\limits^{\infty}_{n=0}C_nx^{n+r}\]
\[(r-p)(r+p)=0\]
3 casos:
\begin{enumerate}
	\item $r_1-r_2\Rightarrow$ no es entero
	\item $r_1-r_2\Rightarrow$ es entero
	\item $r_1=r_2\rightsquigarrow$ p=0
\end{enumerate}
Si $r_1-r_2$ NO ES ENTERO
\[y_1=C_0\Gamma (p+1)\sum\limits^{\infty}_{n=0}\frac{(-1)^k}{k!\Gamma (p+k+1)}\left(\frac{x}{2}\right)^{2k+p}\]
\[\left.J_p(x)=\sum\limits^{\infty}_{n=0}\frac{(-1)^k}{k!\Gamma (p+k+1)}\left(\frac{x}{2}\right)^{2k+p}\right}\parbox{5.5em}{\tiny funcion de bessel de 1er especie de orden p}\]
\[\y=C_1J_p(x+C_2J_{-p}(x))]




\end{document}

