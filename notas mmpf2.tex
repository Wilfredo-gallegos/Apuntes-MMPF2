\documentclass{article}
% Language setting
% Replace `english' with e.g. `spanish' to change the document language
\usepackage[spanish]{babel}

% Set page size and margins
% Replace `letterpaper' with`a4paper' for UK/EU standard size
\usepackage[letterpaper,top=2cm,bottom=2cm,left=1cm,right=1cm,marginparwidth=1.75cm]{geometry}

% Useful packages

\usepackage{verbatim}
\usepackage{amsmath}
\usepackage{amssymb}
\usepackage{amsthm}
\usepackage{amsfonts}
\theoremstyle{definition}
\newtheorem{definition}{Definición}[section]
\newtheorem{theorem}{Teorema}[section]
\newtheorem{corollary}{Corolario}[theorem]
\newtheorem{lemma}{Lema}[section]
\usepackage{etoolbox}
\usepackage{graphicx}
\usepackage{upgreek}
\newcommand{\bigslant}[2]{{\raisebox{.2em}{$#1$}\left/\raisebox{-.2em}{$#2$}\right.}}
\usepackage{xcolor}
\usepackage[colorlinks=true, allcolors=blue]{hyperref}

\title{Metodos matemáticos 2}
\author{Wilfredo Gallegos}

\begin{document}
\maketitle

\textbf{viernes 7 de julio}
\section{Función Gamma}
\textbf{L'imite al infinito(Euler)}\\

\[\begin{array}{rl}
	\Gamma (z)\equiv & \lim\limits_{n\to \infty} \frac{1\cdot 2\cdot 3\cdot\cdot\cdot n}{z(z+1)(z+2)\cdot\cdot\cdot (z+n)}n^z,\ \ \ z\in\mathbb{Z}^{+}\ o\ z\in\mathbb{C}\\
	\Gamma(z+1)=& \lim\limits_{n\to \infty} \frac{1\cdot 2\cdot 3\cdot\cdot\cdot n}{(z+1)(z+2)\cdot\cdot\cdot (z+n+1)}n^{z+1}\\
	=& z\cdot\Gamma (z)= \lim\limits_{n\to \infty}\frac{nz}{z+n+1}\Gamma (z)=\Gamma (z+1)=z\cdot\Gamma (z)
\end{array}\]
Aplicando lo anterior a z=1,2,3...n

\[\begin{array}{rl}

	\Gamma (1)=& \lim\limits_{n\to \infty} \frac{1\cdot 2\cdot 3\cdot\cdot\cdot n}{(1)(2)\cdot\cdot\cdot (n+1)}n^z= \lim\limits_{n\to \infty}\frac{n}{n+1}=1\\
	\Gamma (2)=& z\cdot \Gamma (z)= \Gamma (z+1)=\Gamma (1+1)=1\\
	\Gamma (3)=& 2\cdot \Gamma (1)= \Gamma (2+1)=2\cdot 1\\
	\Gamma (4)=& 3\cdot \Gamma (3)= \Gamma (3+1)=3\cdot 2\cdot 1\\
	.\\
	.\\
	.\\
	\Gamma (n)=&(n-1)!

\end{array}\]
\textbf{Itegral Definida(Integral de Euler)}
\[\Gamma (z)=\int^{\infty}_0 e^{-t}t^{z-1}dt,\ \ \ Re(z)>0\]
Ej. como aparecen en f'isica
\[\Gamma (z)= 2\int^{\infty}_0 e^{-t^2}t^{z+1}dt\ \ \ o \ \ \ \Gamma (z) = \int^1_0\left[ln(\frac{1}{t})\right]^{z-1}dt\]

Si $z=\frac{1}{2}\Rightarrow \Gamma (\frac{1}{2}) = \sqrt{\pi}$ es integral error de Gauss
\[F(z,n)=\int^n_0\left(1-\frac{t}{n}\right)^nt^{z-1}dt,\ \ \ \textit{Re(z)}>\textit{0 con n entero positivo}\ni\]
\[e^{-t}=\lim\limits_{n\to \infty}\left(1-\frac{t}{n}\right)^n\]
\[\Rightarrow F(z,n)=\int^n_0\left(1-\frac{t}{n}\right)^nt^{z-1}dt = \int^1_0 (1-u)^n(un)^{z-1}ndu\ni u=\frac{t}{n}\]
\[\frac{F(z,n)}{n^z}=\int^1_0 (1-u)^n(u)^{z-1}du\]\\

Ahora por integraci'on por partes usando $u=(1-u)^n$, $du=n(1-u)^{n-1}du$, $v=\frac{u^z}{z}$, $dv=u^{z-1}du$ tenemos
\[
\begin{array}{rl}
	\frac{F(z,n)}{n^z}=& (1-u)^n \left.\frac{u^z}{z}\right|^1_0-\int^1_0\frac{u^z}{z} n(1-u)^{n-1}du\\
	F(z,n)=& n^z\cdot \frac{n(n-1)\cdot\cdot\cdot 1}{z(z+1)\cdot\cdot\cdot (z+n-1)}=\int^1_0u^{z+n-1}du\\
	=& \frac{1\cdot 2\cdot\cdot\cdot n}{z(z+1)\cdot\cdot\cdot (z+n)}n^z\Rightarrow\lim\limits_{n\to \infty}F(z,n)=\Gamma (z)
\end{array}\]

\textbf{Producto infinito(Weierstrass)}
\[\frac{1}{\Gamma (z)}= z e^{\delta z} \prod^{\infty}_{n=1}\left(1+\frac{z}{n}\right)e^{-\frac{z}{n}}\]
\begin{center}
	$\delta$ := constante de Euler-Mascheron
	
	$\delta$ := 0.5772156619
\end{center}

\textbf{Ecuaci'on estad'istica de Maxwell-Boltzmann}
\[\begin{array}{lcl}
	&& \textit{K es la constante de Boltzmann}\\
	e^{-E/KT} && \textit{T es la temperatura absoluta}\\
	&& \textit{E(energ'ia): estado de energ'ia ocupada}

\end{array}\]

probabilidad de estar en estado de energ'ia es $Y_{kt}=\beta$

\[\begin{array}{lcl}
	P(E)=Ce^{-\beta E} && \textit{Para un gas idel sin estructura}\\
	n(E)dE && n(E)^{1/2}\\
	1=C\int n(E) e^{-\beta E}dE && \textit{E=energ'ia cin'etica}
\end{array}\]
\par\noindent\rule{\textwidth}{0.5pt}
\[
\begin{array}{rlr}
	1=& c\int^{\infty}_0 E^{1/2}e^{-\beta E}dE=\frac{C\Gamma (3/2)}{\beta^{3/2}},& \beta E = T\Rightarrow dE=\frac{dt}{\beta}\\
	1=& C\int^{\infty}_0 e^{-t}\left(\frac{t}{\beta}\right)^{1/2}\frac{1}{\beta} dt\\
	&= C \int^{\infty}_0 e^{-t} t^{3/2-1}dt\cdot\frac{1}{\beta^{3/2}}\\
	=& \frac{C\Gamma (3/2)}{\beta^{3/2}}\\
	=& c\cdot\frac{\sqrt{\pi}}{2\cdot\beta^{3/2}}
\end{array}\]
\[\therefore C= \frac{2\cdot \beta^{3/2}}{\sqrt{\pi}}\]

\end{document}


