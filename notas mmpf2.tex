\documentclass{article}
% Language setting
% Replace `english' with e.g. `spanish' to change the document language
\usepackage[spanish]{babel}

% Set page size and margins
% Replace `letterpaper' with`a4paper' for UK/EU standard size
\usepackage[letterpaper,top=2cm,bottom=2cm,left=1cm,right=1cm,marginparwidth=1.75cm]{geometry}

% Useful packages

\usepackage{verbatim}
\usepackage{amsmath}
\usepackage{amssymb}
\usepackage[makeroom]{cancel}
\usepackage{amsthm}
\usepackage{amsfonts}
\theoremstyle{definition}
\newtheorem{definition}{Definición}[section]
\newtheorem{theorem}{Teorema}[section]
\newtheorem{corollary}{Corolario}[theorem]
\newtheorem{lemma}{Lema}[section]
\usepackage{etoolbox}
\usepackage{graphicx}
\usepackage{upgreek}
\newcommand{\bigslant}[2]{{\raisebox{.2em}{$#1$}\left/\raisebox{-.2em}{$#2$}\right.}}
\usepackage{xcolor}
\usepackage[colorlinks=true, allcolors=blue]{hyperref}

\title{Metodos matemáticos 2}
\author{Wilfredo Gallegos}

\begin{document}
\maketitle

\textbf{viernes 7 de julio}
\section{Función Gamma}
\textbf{L'imite al infinito(Euler)}\\

\[\begin{array}{rl}
	\Gamma (z)\equiv & \lim\limits_{n\to \infty} \frac{1\cdot 2\cdot 3\cdot\cdot\cdot n}{z(z+1)(z+2)\cdot\cdot\cdot (z+n)}n^z,\ \ \ z\in\mathbb{Z}^{+}\ o\ z\in\mathbb{C}\\
	\Gamma(z+1)=& \lim\limits_{n\to \infty} \frac{1\cdot 2\cdot 3\cdot\cdot\cdot n}{(z+1)(z+2)\cdot\cdot\cdot (z+n+1)}n^{z+1}\\
	=& z\cdot\Gamma (z)= \lim\limits_{n\to \infty}\frac{nz}{z+n+1}\Gamma (z)=\Gamma (z+1)=z\cdot\Gamma (z)
\end{array}\]
Aplicando lo anterior a z=1,2,3...n

\[\begin{array}{rl}

	\Gamma (1)=& \lim\limits_{n\to \infty} \frac{1\cdot 2\cdot 3\cdot\cdot\cdot n}{(1)(2)\cdot\cdot\cdot (n+1)}n^z= \lim\limits_{n\to \infty}\frac{n}{n+1}=1\\
	\Gamma (2)=& z\cdot \Gamma (z)= \Gamma (z+1)=\Gamma (1+1)=1\\
	\Gamma (3)=& 2\cdot \Gamma (1)= \Gamma (2+1)=2\cdot 1\\
	\Gamma (4)=& 3\cdot \Gamma (3)= \Gamma (3+1)=3\cdot 2\cdot 1\\
	.\\
	.\\
	.\\
	\Gamma (n)=&(n-1)!

\end{array}\]
\textbf{Itegral Definida(Integral de Euler)}
\[\Gamma (z)=\int^{\infty}_0 e^{-t}t^{z-1}dt,\ \ \ Re(z)>0\]
Ej. como aparecen en f'isica
\[\Gamma (z)= 2\int^{\infty}_0 e^{-t^2}t^{z+1}dt\ \ \ o \ \ \ \Gamma (z) = \int^1_0\left[ln(\frac{1}{t})\right]^{z-1}dt\]

Si $z=\frac{1}{2}\Rightarrow \Gamma (\frac{1}{2}) = \sqrt{\pi}$ es integral error de Gauss
\[F(z,n)=\int^n_0\left(1-\frac{t}{n}\right)^nt^{z-1}dt,\ \ \ \textit{Re(z)}>\textit{0 con n entero positivo}\ni\]
\[e^{-t}=\lim\limits_{n\to \infty}\left(1-\frac{t}{n}\right)^n\]
\[\Rightarrow F(z,n)=\int^n_0\left(1-\frac{t}{n}\right)^nt^{z-1}dt = \int^1_0 (1-u)^n(un)^{z-1}ndu\ni u=\frac{t}{n}\]
\[\frac{F(z,n)}{n^z}=\int^1_0 (1-u)^n(u)^{z-1}du\]\\

Ahora por integraci'on por partes usando $u=(1-u)^n$, $du=n(1-u)^{n-1}du$, $v=\frac{u^z}{z}$, $dv=u^{z-1}du$ tenemos
\[
\begin{array}{rl}
	\frac{F(z,n)}{n^z}=& (1-u)^n \left.\frac{u^z}{z}\right|^1_0-\int^1_0\frac{u^z}{z} n(1-u)^{n-1}du\\
	F(z,n)=& n^z\cdot \frac{n(n-1)\cdot\cdot\cdot 1}{z(z+1)\cdot\cdot\cdot (z+n-1)}=\int^1_0u^{z+n-1}du\\
	=& \frac{1\cdot 2\cdot\cdot\cdot n}{z(z+1)\cdot\cdot\cdot (z+n)}n^z\Rightarrow\lim\limits_{n\to \infty}F(z,n)=\Gamma (z)
\end{array}\]

\textbf{Producto infinito(Weierstrass)}
\[\frac{1}{\Gamma (z)}= z e^{\delta z} \prod^{\infty}_{n=1}\left(1+\frac{z}{n}\right)e^{-\frac{z}{n}}\]
\begin{center}
	$\delta$ := constante de Euler-Mascheron
	
	$\delta$ := 0.5772156619
\end{center}

\textbf{Ecuaci'on estad'istica de Maxwell-Boltzmann}
\[\begin{array}{lcl}
	&& \textit{K es la constante de Boltzmann}\\
	e^{-E/KT} && \textit{T es la temperatura absoluta}\\
	&& \textit{E(energ'ia): estado de energ'ia ocupada}

\end{array}\]

probabilidad de estar en estado de energ'ia es $Y_{kt}=\beta$

\[\begin{array}{lcl}
	P(E)=Ce^{-\beta E} && \textit{Para un gas idel sin estructura}\\
	n(E)dE && n(E)^{1/2}\\
	1=C\int n(E) e^{-\beta E}dE && \textit{E=energ'ia cin'etica}
\end{array}\]
\par\noindent\rule{\textwidth}{0.5pt}
\[
\begin{array}{rlr}
	1=& c\int^{\infty}_0 E^{1/2}e^{-\beta E}dE=\frac{C\Gamma (3/2)}{\beta^{3/2}},& \beta E = T\Rightarrow dE=\frac{dt}{\beta}\\
	1=& C\int^{\infty}_0 e^{-t}\left(\frac{t}{\beta}\right)^{1/2}\frac{1}{\beta} dt\\
	&= C \int^{\infty}_0 e^{-t} t^{3/2-1}dt\cdot\frac{1}{\beta^{3/2}}\\
	=& \frac{C\Gamma (3/2)}{\beta^{3/2}}\\
	=& c\cdot\frac{\sqrt{\pi}}{2\cdot\beta^{3/2}}
\end{array}\]
\[\therefore C= \frac{2\cdot \beta^{3/2}}{\sqrt{\pi}}\]
\newpage

\textbf{lunes 10 de julio}\\
\textbf{Relaciones de funciones}

\[\Gamma (z+1) = z\Gamma (z)\]
\textbf{Fomula de reflexi'on}

\[\Gamma (z)\Gamma (1-z) = \frac{\pi}{sen(z\pi)}\]
\textbf{F'ormuláde duplicaci'on de Legendre}

\[\Gamma (1+z)\Gamma (z+\frac{1}{2}) = 2^{-2z}\sqrt{\pi}\Gamma (2z+1) \]
\textbf{Notac'i'on doble factorial: }
\[
\begin{array}{rl}
	(2n+1)!!=& 1\cdot 3\cdot 5\cdot\cdot\cdot (2n+1)=\frac{(2n+1)!}{2^nn}\\
	(2n)!!=& 2\cdot 4\cdot 6\cdot\cdot\cdot (2n)=2^nn!\\
	(-1)!!=& 1
\end{array}
\]
\textbf{PROPIEDADES ANAL'ITICAS}

$[\Gamma (z)]^{-1}$ tiene singularidades en $z=0,-1,-2...$ y no tiene cuando  $z=-1,2,3...$ y no tiene ceros en el plano complejo finito positivo.\\
El residuo $R_n$ cuando $z=-n$ donde n es un entero mayor o igual a cero.\\
\[
\begin{array}{rl}
	R_n=&\lim\limits_{n\to 0} (\epsilon\Gamma (-n+\epsilon))\\
	=& \lim\limits_{n\to 0} \frac{\epsilon\Gamma (-n+1+\epsilon)}{-n+\epsilon}\\
	=& \lim\limits_{n\to 0}\frac{\epsilon\Gamma (-n+2+\epsilon)}{(\n+\epsilon)(\n+1+\epsilon)}\\
	=&\lim\limits_{n\to 0} \frac{\epsilon\Gamma (1+\epsilon)}{(-n+\epsilon)\cdot\cdot\cdot(\epsilon))}\\
	=& \frac{(-1)^n}{n!}, \textit{el residuo alterna signos en } z=-n
\end{array}
\]

\textbf{Integral de Schaefli}
\[\int_Ce^{-t}t^vdt=(e^{2\pi iv})\Gamma (v+1)\]
donde C es el contorno
\begin{center}
\includegraphics[scale=0.4]{../grafico 1.png} 
\end{center}

Esta integral es 'util cuando V no es entero.\\
\textbf{Notaci'on factorial:}
\[\prod (z)=z!=\Gamma (z+1)\]
\textbf{Funci'on Digamma}

\[
\begin{array}{rl}
	\ln(\Gamma (z+1))=& z\cdot\Gamma (z)\\
	=& \lim\limits_{n\to \infty}\frac{z}{z} \ln\left[\frac{n!\cdot n^z}{(z+1)(z_2)\cdot\cdot\cdot (z+n)\right]}\\
	\frac{d}{dz} \ln(\Gamma (z+1))=& \frac{d}{dz}\lim\limits_{n\to \infty}\left[\ln(n!)+z\cdot \ln(n)-\ln(z+1)-\ln(z+2)-...\ln(z+n)\right]\\
	\frac{d}{dz}\ln (\Gamma (z+1))\equiv \underbrace{\varphi (z+1)}_{Funci'on\ digamma}=& \lim\limits_{n\to \infty}\left( \ln (n)-\frac{1}{z+1}-\frac{1}{z+2}\cdot\cdot\cdot -\frac{1}{z+n}\right)\\
	&\\
	\varphi (z+1)=& \frac{[\Gamma (z+1)]^'}{\Gamma (z+1)}
\end{array}
\]
Si sumamos y restamos un n'umero arm'onico
\[H_n=\sum^n_{m=1}\frac{1}{m}\]
\[\varphi (z+1) =\lim\limits_{n\to \infty} \underbrace{\left[(\ln (n)-H_n)}_{=-\delta\textit{ cuando } n\to\infty}-\sum\left(\frac{1}{z+m}-\frac{1}{m\right)}\right]\]
\[\Rightarrow \varphi (z+1) = -\delta +\sum^{\infty}_{m=1}\frac{z}{m(m+z)},\ \ \ \textit{si z=0}\Rightarrow\varphi (1)=-\delta\]
para $n>0$, n entero.
\[
\begin{array}{rl}
	\varphi (n+1) =& -\delta +\sum^n_{m=1}\frac{1}{m}\\
	=& -\delta+H_n
\end{array}
\]
\textbf{Funci'on Poligamma}

Cuando se deriva muchas veces la funci'on digamma
\[
\begin{array}{rl}
	\varphi^{(m)} (z+1) \equiv& \frac{d^{m+1}}{dz^{m+1}}\ln\Gamma (z+1)\\
	\varphi^{(m)} (z+1)=& (-1)^{m+1}m!\sum^{\infty}_{n=1}\frac{1}{(z+n)^{m+1}},\ \ \ m=1,2,3...
\end{array}
\]
Si z=0 la funci'on se conoce como zeta de Riemann
\[\zeta (m) = \sum^{\infty}_{n-1}\frac{1}{n^m}\]
\[\Rightarrow \varphi^{(m)}_{(1)}=(-1)^{m+1}\cdot m!\cdot \zeta (m+1),\ \ \ m=1,2,3...\]
\textbf{Expansi'on de Maclaurin}
\[
\begin{array}{rl}
	\ln\Gamma (z+1) =& \sum^{\infty}_{n=1}\frac{z^n}{n!}\varphi^{(n-1)}_{(1)}\\
	=& -\delta\cdot z +\sum^{\infty}_{n=2}(-1)^n\frac{z^n}{n}\zeta (n)
\end{array}
\]
converge en $|z|<1$ para $z=x\Rightarrow$ se puede calcular $\Gamma (z+1)$ para n'umeros reales o complejos.\\
\textbf{Suma de series}

Transformar la serie por medio de fracciones parciales y expresar la serie infinita como sumas finitas de funciones gamma y poligamma.\\
\textbf{EJEMPLO: }
\[
\begin{array}{rl}
	\varphi ^{(m)} (z+2)=& \varphi^{(m)} (z+1)+(-1)^{m}\frac{m!}{(z+1)^{m+1}}\\
	\varphi^{(m)} (z+2)-\varphi^{(m)} (z+1)=& \frac{(-1)^mm!}{(z+1)^{m+1}}\\
	\\
	\frac{d^{(m)}}{dz^{(m)}}\left[\varphi (z+2) -\varphi (z+1)\right]\\
	\\
	\varphi (z+2) - \varphi (z+1)=&\sum^{\infty}_{m=1}\frac{z+1}{m(m+z+1)}-\sum^{\infty}_{m=1}\frac{z}{m(m+z)}\\
	=& \sum^{\infty}_{m=1}\left[\frac{1}{m}-\frac{1}{m+z+1}-\frac{1}{m}+\frac{1}{m+z}\right]\\
	=& \frac{1}{z+1}
\end{array}
\]
\[\frac{d^m}{dz^m}\left[\varphi (z+2)-\varphi (z+1)\right] = \frac{d^m}{dz^m}\left[\frac{1}{z+1}\right]=\frac{(-1)^mm!}{(z+1)^{m+1}}\]
Demuestre  que:
\[\frac{1}{2}\ln \left[\frac{\pi\cdot z}{sen(\pi\cdot z)}\right]=\sum^{\infty}_{n=1}\frac{\zeta (2n)}{2n}z^{2n},\ \ \ |z|<1\]
\textcolor{red}{
\begin{itemize}
	\item $\frac{1}{2}\ln\left[\textcolor{green}{\underbrace{\textcolor{red}{\Gamma (z)}}_{z\Gamma (z)=\Gamma (z+1)}}\Gamma (z-1)\right]=\frac{1}{2}\ln\left[\frac{\pi}{sen(z\pi)}\right]$
	\item $\ln \Gamma (z+1) =\lim\limits_{n\to \infty}\left[\ln (n!)+z\ln (n)-\ln (z+1)-...-\ln (z+n)\right]$
\end{itemize}
}
\[\Gamma (z)\Gamma (1-z)\rightsquigarrow\Gamma (1+z)\Gamma (1-z)\]
\[
\begin{array}{rl}
	\ln\left[\Gamma (1+z)\Gamma (1-z)\right]=& -\delta\cdot z +\sum^{\infty}_{n=2}(-1)^n\frac{z^n}{n}\zeta (n)\\
	&+(-\delta(-z))+\sum^{\infty}_{n=2}(-1)^n\frac{(-z)^n}{n}\zeta (n)\\
	=&\sum^{\infty}_{n=2}\zeta (n)\underbrace{\left[\frac{z^n}{n}+\frac{(-z)^n}{n}\right]}_{\textcolor{red}{z^{2n}}}\\
	=& 2\sum^{\infty}_{n=1}\cancelto{1}{(-1)^{2n}}\zeta (2n)\frac{z^{2n}}{2n}
\end{array}
\]
\newpage
\textbf{viernes 14 de julio}

\textbf{La Funci'on Beta}

\[\Gamma (p+q)\propto \Gamma (p)\Gamma (q)\]
\[
\begin{array}{rl}
	n!= \Gamma (n+1)\& \int^{\infty}_0 e^{-t}t^2 dt,\ \ \textit{Variante de gamma para definir la funi'on beta}\\
	\Gamma (m)\Gamma (n)=& \int^{\infty}_0 e^{-u}u^m fu\int^{\infty}_0 e^{-v}v^n dv
\end{array}
\]
\[
\begin{array}{cc}
	u=x^2 & v=y^2\\
	du=2xdx & dv= 2ydy
\end{array}
\]
\[
\begin{array}{rl}
	\Gamma (m)\Gamma (n) =& \int^{\infty}_0 \int^{\infty}_0e^{-x^{2}}e^{-y^{2}}x^{2m}(2x)y^{2m}(2y)dxdy\\
	=& 4\int^{\infty}_0 e^{-x^{2}} e^{-x^{2}}e^{-y^{2}}x^{2m+1}y^{2m+1}dxdy
\end{array}
\]
otro cambio de variables
\[
\begin{array}{cc}
	x=rcos(\theta) & y=rsen(\theta)
\end{array}
\]
Usando el jacobiano
\[
\left|
\begin{array}{ccc}
	\frac{\partial x}{\partial r}= cos\theta dr && \frac{\partial x}{\partial \theta} =-rsen\theta d\theta\\
	\\
	\frac{\partial y}{\partial r} = sen\theta dr && \frac{\partial y}{\partial \theta} r cos\theta d\theta
\end{array}\right|
=rdrd\theta
\]
\[
\begin{array}{rl}
	=& \int^{\infty}_0\int^{\pi/2}_0 \cancelto{e^{-r^2cos^2\theta}}{e^{-r^2}} r^{2m+2n+3} cos^{2m+1}\theta sen ^{2n+1}drd\theta\\
	\\Gamma (m)\Gamma (n) =& 2 \Gamma (m+n+1)\int^{\pi/2}_0 cos^{2m+1}\theta sen^{2n+1}\theta d\theta
\end{array}
\]
\[
\begin{array}{rl}
	\Gamma (z)=& 2 \int^{\infty}_0 e^{-t^2}t^{2z-1}dt;\ Re(z>0)\\
	\Gamma (q)\Gamma (p) =& 4\int^{\infty}_0 e^{-s^2} s^{2q-1}ds\int^{\infty}_0 e^{-t^2}t^{2p-1}dt
\end{array}
\]
\[
\begin{array}{cc}
	s=rcos\theta & t=rsen\theta\\
	r^2=s^2+t^2 & dsdt=rdrd\theta
\end{array}
\]
\[
\begin{array}{rl}
	\Gamma (q)\Gamma (p)=& 4\int\int e^{-r^2cos^2\theta}e^{-r^2sen^2\theta}r^{2q-1}cos^{2q-1}r^{2p-1}sen^{2p-1}rdrd\theta\\
	=& 2\cdot 2 \int^{\infty}_0 e^{-r^2}r^{2q+2p-1}cos^{2q-1}sen^{2p-1}drd\theta\\
	\textcolor{blue}{\Gamma (p)\Gamma (q)=}&\textcolor{blue}{2\int cos^{2q-1}sen^{2p-1} d\theta}
\end{array}
\]
\[\frac{m! n!}{(m+n+1)!}=2\int^{pi /2}_0 cos^{2m+1}\theta sen^{2n+1}\theta d \theta\]

\begin{center}
\fbox{\begin{minipage}{15em}
  \[b(p,q)=\frac{\Gamma (q)\Gamma (p)}{\Gamma (q+p)}\]
\end{minipage}} definici'on de funci'on beta
\end{center}
\[b(p,q)=b(q,p)\]
Si se sustituye $t=cos^2\theta$ y $dt=-2 2cos\theta sen\theta d\theta$




\end{document}


